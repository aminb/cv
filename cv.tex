%% cv.tex
%% Copyright 2016-2017 Amin Bandali <amin@aminb.org>
%
% This work may be distributed and/or modified under the
% conditions of the LaTeX Project Public License, either version 1.3
% of this license or (at your option) any later version.
% The latest version of this license is in
%   http://www.latex-project.org/lppl.txt
% and version 1.3 or later is part of all distributions of LaTeX
% version 2005/12/01 or later.
%
% Based on Jason R. Blevins's Curriculum Vitae template,
% Copyright (C) 2004-2016 Jason R. Blevins <jrblevin@sdf.org>
% http://jblevins.org/

\documentclass[12pt,letterpaper]{article}

\usepackage{hyperref}
\usepackage{geometry}
\usepackage{enumitem}

% Fonts
\usepackage{graphicx}
\usepackage{fontspec}
\setmainfont[Numbers=Lining]{EB Garamond}
\setmonofont[Scale=0.85]{Ubuntu Mono}
\newfontfamily{\smallcaps}[RawFeature={+c2sc,+scmp}]{EB Garamond}
\newcommand{\amper}{{\fontspec[Scale=.9]{EB Garamond}\selectfont\itshape\&}}

%% Load Microtype with default settings. This will use the
%% EB-Garamond protrusion definitions if present.
\usepackage{microtype}


\def\name{Amin Bandali}

% The following metadata will show up in the PDF properties
\hypersetup{
  colorlinks = true,
  urlcolor = black,
  pdfauthor = {\name},
  pdfkeywords = {Programming Languages, Haskell, Rust, Formal Methods, Type
    Systems, Proof Systems, Automated Provers},
  pdftitle = {\name: Curriculum Vitae},
  pdfsubject = {Curriculum Vitae},
  pdfpagemode = UseNone
}

\geometry{
  body={6.5in, 9.0in},
  left=1.0in,
  top=1.0in
}

% Customize page headers
\pagestyle{myheadings}
\markright{\name}
\thispagestyle{empty}

% Custom section fonts
\usepackage{sectsty}
\sectionfont{\rmfamily\mdseries\Large}
\subsectionfont{\rmfamily\mdseries\itshape\large}

% Other possible font commands include:
% \ttfamily for teletype,
% \sffamily for sans serif,
% \bfseries for bold,
% \scshape for small caps,
% \normalsize, \large, \Large, \LARGE sizes.

% Don't indent paragraphs.
\setlength\parindent{0em}

% Make lists without bullets and compact spacing
\renewenvironment{itemize}{
  \begin{list}{}{
    \setlength{\leftmargin}{1.5em}
    \setlength{\itemsep}{0.25em}
    \setlength{\parskip}{0pt}
    \setlength{\parsep}{0.25em}
  }
}{
  \end{list}
}
\setlist[enumerate]{itemsep=0.25em}

\begin{document}

% Place name at left
{\huge \name}

% Alternatively, print name centered and bold:
%\centerline{\huge \bf \name}

\bigskip

\begin{minipage}[t]{0.495\textwidth}
  Email: \href{mailto:amin9@my.yorku.ca}{amin9@my.yorku.ca} \\
  Homepage: \href{https://aminb.org}{https://aminb.org} \\
  Phone: available upon request
\end{minipage}

\section*{Research Interests}

\begin{itemize}
\item Functional Programming and functional languages.
\item Formal methods, especially type systems, proof systems, and automated
  provers.
\item Verification, Haskell, and Rust.
\end{itemize}

\section*{Education}

\begin{itemize}
  \setlength\itemsep{.75em}
\item {\large B.Sc. Honours Computer Science}\, |\, {\small 2013--present}

  \textit{York University}, Toronto, Canada

  \begin{itemize}
  \item Expected completion: December 2017 \hspace{1em} GPA: 7.9/9.0
  \item Relevant courses: System Specification \amper\
    Refinement, Software Requirements Eng., Software Design, Operating
    Systems, Computational Complexity, Design \amper\ Analysis of Algorithms.
  \item Finished first year (2013-14) at \textit{Carleton University} with a GPA
    of 11.0/12.0 then transferred to \textit{York University} in Fall 2014.
  \end{itemize}

%% \item {\large High School Diploma}\, |\, {\small 2013}

%%   \textit{Glebe Collegiate Institute}, Ottawa, Canada

%%   \hspace{1.3em} Average: 94.3\%
\end{itemize}

\section*{Research Experience}

\begin{itemize}
\item {\large Software Engineering Lab, } York University\, |\, {\small summer
  2017}

  \textit{Research Assistant}

  \begin{itemize}
  \item I'm working on expanding and testing the \texttt{mathmodels} library,
    collection of abstract mathematical collections written in Eiffel. I'll also
    be modeling various pieces of software in TLA+.
  \end{itemize}

\item {\large Software Engineering Lab, } York University\, |\, {\small summer 2016}

  \textit{Research Student}

  \begin{itemize}
  \item As an undergraduate research student, I worked on \textit{Literate
      Unit-B}, the verifier for Unit-B, a new formal method focused on formal
    verification of reactive, concurrent and distributed systems.

    From the Literate Unit-B codebase (written in Haskell), I decoupled the
    logic module and used it to build \textit{Unit-B Web}, a web interface using
    Literate Unit-B to do predicate calculus proofs. \linebreak Unit-B Web, also
    written in Haskell, supports the \LaTeX\ syntax of the Unit-B logic, renders
    user input on the page, and calls the sequent prover of the logic module,
    which uses the Z3 SMT solver to check the validity of user input.

  \item Further, I separated Literate Unit-B's type checker from its parser,
    allowing easier substitution of other type checking algorithms and in
    preparation for implementing subtyping.
  \end{itemize}
\end{itemize}

\section*{Professional Experience}
\begin{itemize}
\item {\large Lotek Wireless Inc., } Newmarket, Canada\, |\, {\small 2015--2016}

  \textit{Software Developer}

  \begin{itemize}
  \item Designed and implemented various applications in C\# and C to test and
    analyze a satellite pass prediction algorithm for predicting the pass
    windows of Argos satellites, for scheduling send times of data collected by
    company's wildlife tracking products.

  \item Designed and developed an Employee Portal web application in C\# and the
    MVC framework, used by employees for accessing various data catalogs and
    archives.
  \end{itemize}
  \vspace{.25em}

\item {\large Athlete Builder, } Ottawa, Canada\, |\, {\small 2013--2014}

  \textit{Software Developer}

  \begin{itemize}
  \item Developed the Backend of Athlete Builder platform in C\# and MVC.
  \item Was a key role in development of the platform core.
  \item Developed the alpha version of Athlete Builder Android app in Java.
  \end{itemize}
\end{itemize}

\section*{Volunteer Activities}

\begin{itemize}

\item {\large EmacsConf 2015, } \href{http://emacsconf.org}{emacsconf.org}\, |\,
  {\small summer 2015}

  \textit{Organizer}

  \begin{itemize}
  \item EmacsConf is a conference about the joy of Emacs and writing Emacs
    Lisp. I was a key organizer and in charge of setting up and maintaining
    several vital pieces of the EmacsConf infrastructure.
  \end{itemize}
  \vspace{.25em}

\item {\large VONICAL Inc., } Ottawa, Canada\, |\, {\small spring 2013}

  \textit{Application Developer}

  \begin{itemize}
  \item As a volunteer, worked on development of EARN (Employment Accessibility
    Resource Network) portal using the Anahita social networking platform, in
    PHP under Linux.
  \end{itemize}
  \vspace{.25em}

\item {\large Hire Works Inc., } Ottawa, Canada\, |\, {\small winter 2013}

  \textit{Mobile \& Web Developer}

  \begin{itemize}
  \item As a volunteer, I worked on a variety of web and mobile projects for
    Hire Works, Inc.
  \end{itemize}
  \vspace{.25em}

\item {\large St. Brigid's Summer Camp, } Ottawa, Canada\, |\, {\small summer
    2012}

  \textit{Web Developer}

  \begin{itemize}
  \item As a volunteer, I re-designed and coded (from scratch) an updated and
    revamped version of the photo gallery section of St. Brigid Summer Camp's
    website in PHP and JavaScript. A refactored version of my code is deployed
    and being used.
  \end{itemize}
  % \vspace{.25em}

\end{itemize}


\section*{Recent Projects}

\begin{itemize}
\item \textit{Unit-B Web:} The web interface for Unit-B, as mentioned in the
  \textit{Research Experience} section.\\
  Source code available at
  \href{https://github.com/unitb/unitb-web}{https://github.com/unitb/unitb-web}

\item \textit{tex2png-hs:} A tool for easily converting \TeX\ and \LaTeX\ to PNG
  images. \verb#tex2png-hs# is a Haskell port of Xyne's \verb#tex2png# tool. It
  is a wrapper around \verb#latex# and \verb#dvipng# and provides several
  options for modifying its behaviour, such as cropping the whitespace around
  the content, specifying the DPI, or inputting a full document.\\
  Source code available at
  \href{https://github.com/unitb/tex2png-hs}{https://github.com/unitb/tex2png-hs}

\item For more projects, visit my GitHub profile at
  \href{https://github.com/aminb}{https://github.com/aminb}.
\end{itemize}

\section*{Miscellaneous}

\begin{itemize}
\item \textit{Programming Languages:} Haskell, Rust, Eiffel, Python, C, Emacs
  Lisp, C\#, JavaScript.
\item \textit{Platforms:} Arch Linux, Ubuntu and other distros, Android, macOS,
  Windows.
\item \textit{Tools:} Emacs, Git, Zsh, \LaTeX, CI Systems (e.g. Travis CI),
  Rodin, SQL DBs.
\item \textit{Languages:} Persian (mother tongue), English (fluent), French
  (beginner).
\end{itemize}


% Footer
\bigskip
{\small Last updated: \today}

\end{document}
